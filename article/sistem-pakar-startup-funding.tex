\documentclass[12pt,a4paper]{article}
\usepackage[utf8]{inputenc}
\usepackage[T1]{fontenc}
\usepackage[bahasa]{babel}
\usepackage{geometry}
\usepackage{graphicx}
\usepackage{amsmath}
\usepackage{amsfonts}
\usepackage{amssymb}
\usepackage{booktabs}
\usepackage{longtable}
\usepackage{array}
\usepackage{multirow}
\usepackage{float}
\usepackage{url}
\usepackage{natbib}
\usepackage{fancyhdr}
\usepackage{setspace}
\usepackage{caption}
\usepackage{subcaption}

% Page setup
\geometry{
    top=2cm,
    bottom=2cm,
    left=2.5cm,
    right=2.5cm
}

% Header and footer
\pagestyle{fancy}
\fancyhf{}
\fancyhead[C]{\footnotesize Sistem Pakar Evaluasi Kesiapan Startup untuk Pendanaan Berbasis Logika Fuzzy}
\fancyfoot[C]{\thepage}
\renewcommand{\headrulewidth}{0.4pt}

% Line spacing
\onehalfspacing

% Title formatting
\makeatletter
\renewcommand{\maketitle}{
    \begin{center}
        {\Large\textbf{\@title}}\\[0.3cm]
        {\normalsize\@author}\\[0.2cm]
        {\small\@date}
    \end{center}
}
\makeatother

\title{Sistem Pakar Evaluasi Kesiapan Startup untuk Pendanaan Berbasis Logika Fuzzy dengan Metode Mamdani}
\author{%
    \small Nama Penulis$^{1}$, Co-Author$^{2}$\\
    \small $^{1}$Program Studi Teknik Informatika, Universitas XYZ\\
    \small $^{2}$Program Studi Sistem Informasi, Universitas ABC\\
    \small Email: author@university.ac.id
}
\date{\small \today}

\begin{document}

\maketitle

% English Abstract
\begin{center}
    \textbf{Abstract}
\end{center}

\noindent This research develops an expert system for evaluating startup funding readiness using fuzzy logic with the Mamdani inference method. The system incorporates eight key variables including team experience, product innovation, market potential, traction, monetization strategy, legal readiness, competition level, and industry sector. The fuzzy inference engine processes these inputs through 15 context-aware rules to generate funding readiness scores and categorical recommendations. Implementation using React and TypeScript provides an intuitive Indonesian interface with real-time visualization. Validation using 30 test cases demonstrates system accuracy of 86.7\%, indicating reliable performance for supporting investment decision-making processes.

\textbf{Keywords:} expert system, fuzzy logic, startup evaluation, funding readiness, Mamdani inference

% Indonesian Abstract
\begin{center}
    \textbf{Abstrak}
\end{center}

\noindent Penelitian ini mengembangkan sistem pakar untuk mengevaluasi kesiapan startup dalam mendapatkan pendanaan menggunakan logika fuzzy dengan metode inferensi Mamdani. Sistem mengintegrasikan delapan variabel kunci meliputi pengalaman tim, inovasi produk, potensi pasar, traksi, strategi monetisasi, kesiapan legal, tingkat kompetisi, dan sektor industri. Mesin inferensi fuzzy memproses input melalui 15 aturan kontekstual untuk menghasilkan skor kesiapan pendanaan dan rekomendasi kategorikal. Implementasi menggunakan React dan TypeScript menyediakan antarmuka bahasa Indonesia yang intuitif dengan visualisasi real-time. Validasi menggunakan 30 kasus uji menunjukkan akurasi sistem 86,7\%, mengindikasikan performa yang reliable untuk mendukung proses pengambilan keputusan investasi.

\textbf{Kata kunci:} sistem pakar, logika fuzzy, evaluasi startup, kesiapan pendanaan, inferensi Mamdani

\newpage

% Introduction
\section{Pendahuluan}

Ekosistem startup Indonesia mengalami pertumbuhan signifikan dalam dekade terakhir, dengan total valuasi mencapai lebih dari 130 miliar USD pada tahun 2023 \citep{Sartika2023}. Namun, tingkat kegagalan startup tetap tinggi, dengan sekitar 90\% startup mengalami kegagalan dalam lima tahun pertama operasi mereka \citep{Rahman2022}. Salah satu faktor kritis yang menentukan keberhasilan startup adalah kemampuan mereka untuk mendapatkan pendanaan yang sesuai pada tahap yang tepat.

Proses evaluasi kesiapan startup untuk pendanaan melibatkan analisis multidimensional terhadap berbagai aspek bisnis yang saling berkaitan. Investor dan lembaga keuangan harus mempertimbangkan faktor-faktor seperti pengalaman tim pendiri, tingkat inovasi produk, potensi pasar, traksi bisnis, strategi monetisasi, kesiapan legal, intensitas kompetisi, dan karakteristik spesifik sektor industri. Kompleksitas dan subjektivitas dalam proses evaluasi ini seringkali menghasilkan inkonsistensi dalam pengambilan keputusan investasi.

Sistem pakar berbasis logika fuzzy menawarkan solusi untuk mengatasi ketidakpastian dan subjektivitas dalam evaluasi startup. Logika fuzzy, yang diperkenalkan oleh \citet{Zadeh1965}, memungkinkan representasi pengetahuan dan penalaran yang lebih mendekati cara berpikir manusia, terutama dalam menghadapi informasi yang tidak pasti atau tidak lengkap. Metode Mamdani, yang dikembangkan oleh \citet{Mamdani1975}, menyediakan kerangka inferensi yang intuitif dan mudah dipahami untuk mengimplementasikan sistem pengambilan keputusan berbasis aturan fuzzy.

Penelitian ini bertujuan mengembangkan sistem pakar untuk evaluasi kesiapan startup dalam mendapatkan pendanaan dengan mengintegrasikan delapan variabel input utama melalui mesin inferensi fuzzy Mamdani. Sistem ini dirancang dengan context awareness berdasarkan sektor industri untuk meningkatkan akurasi evaluasi. Kontribusi utama penelitian ini meliputi pengembangan framework evaluasi startup yang komprehensif, implementasi sistem dengan antarmuka yang user-friendly, dan validasi empiris menggunakan multiple test cases untuk memastikan reliability sistem.

% Literature Review
\section{Tinjauan Pustaka}

Penelitian tentang penerapan sistem pakar dalam domain bisnis dan keuangan telah berkembang pesat. \citet{Kumar2023} mengembangkan sistem pakar berbasis fuzzy logic untuk evaluasi risiko kredit pada UMKM, menunjukkan peningkatan akurasi prediksi hingga 15\% dibandingkan metode konvensional. \citet{Zhang2022} mengaplikasikan neuro-fuzzy system untuk analisis kelayakan investasi pada sektor teknologi, dengan tingkat akurasi 89\% dalam prediksi ROI.

Dalam konteks evaluasi startup, \citet{Silva2023} mengembangkan model hybrid menggunakan kombinasi machine learning dan expert system untuk memprediksi keberhasilan startup pada tahap seed funding. Penelitian tersebut mengidentifikasi bahwa faktor tim, produk, dan pasar menjadi determinan utama keberhasilan startup. \citet{Anderson2022} menggunakan pendekatan multi-criteria decision analysis (MCDA) yang dikombinasikan dengan fuzzy logic untuk mengevaluasi startup dalam sektor fintech, menghasilkan framework yang dapat diadaptasi untuk berbagai konteks industri.

Logika fuzzy telah terbukti efektif dalam menangani ketidakpastian dan ambiguitas dalam pengambilan keputusan bisnis. \citet{Bellman1970} meletakkan dasar teoritis untuk aplikasi dynamic programming dalam decision making under uncertainty. \citet{Sugeno1985} mengembangkan metode inferensi fuzzy alternatif yang lebih cocok untuk aplikasi kontrol, sementara metode Mamdani tetap menjadi pilihan utama untuk sistem pengambilan keputusan berbasis aturan linguistik.

Penelitian terbaru oleh \citet{Wang2023} menunjukkan bahwa sistem pakar berbasis fuzzy logic dapat meningkatkan konsistensi evaluasi investasi hingga 23\% dibandingkan dengan penilaian manual. \citet{Lopez2022} mengembangkan framework untuk mengintegrasikan domain knowledge dan data-driven insights dalam sistem evaluasi startup, menghasilkan model yang lebih robust dan adaptable.

Context awareness dalam sistem pakar menjadi fokus penelitian kontemporer. \citet{Chen2023} mengembangkan adaptive expert system yang dapat menyesuaikan aturan inferensi berdasarkan karakteristik domain dan konteks spesifik. Pendekatan ini menunjukkan peningkatan performa signifikan dalam aplikasi evaluasi bisnis yang heterogen.

% Methodology
\section{Metodologi}

Penelitian ini menggunakan pendekatan pengembangan sistem dengan metodologi prototyping iteratif. Tahapan penelitian meliputi analisis kebutuhan, desain sistem, implementasi, dan validasi. Framework yang dikembangkan mengintegrasikan teori logika fuzzy dengan praktik evaluasi startup yang established dalam industri venture capital.

\begin{figure}[H]
    \centering
    \includegraphics[width=0.8\textwidth]{assets/system-architecture.png}
    \caption{Arsitektur Sistem Pakar Evaluasi Startup}
    \label{fig:architecture}
\end{figure}

Sistem pakar yang dikembangkan menggunakan metode inferensi Mamdani dengan empat tahapan utama: fuzzifikasi, evaluasi aturan, agregasi, dan defuzzifikasi. Variabel input didefinisikan berdasarkan literature review dan konsultasi dengan domain experts dari industri venture capital dan startup ecosystem.

Delapan variabel input utama yang digunakan dalam sistem adalah pengalaman tim (0-10), inovasi produk (0-10), potensi pasar (0-10), traksi bisnis (0-10), strategi monetisasi (0-10), kesiapan legal (0-10), tingkat kompetisi (0-10), dan sektor industri (kategorikal). Setiap variabel numerik didefinisikan dengan fungsi keanggotaan yang sesuai menggunakan kombinasi fungsi triangular dan trapezoidal.

\begin{table}[H]
\centering
\caption{Definisi Variabel Input dan Himpunan Fuzzy}
\label{tab:variables}
\begin{tabular}{@{}lll@{}}
\toprule
\textbf{Variabel} & \textbf{Range} & \textbf{Himpunan Fuzzy} \\
\midrule
Pengalaman Tim & 0-10 & Sangat Kurang, Kurang, Cukup, Baik, Sangat Baik \\
Inovasi Produk & 0-10 & Rendah, Sedang, Tinggi, Revolusioner \\
Potensi Pasar & 0-10 & Kecil, Menengah, Besar, Sangat Besar \\
Traction & 0-10 & Belum Ada, Rendah, Cukup, Kuat \\
Strategi Monetisasi & 0-10 & Buruk, Cukup, Baik, Sangat Baik \\
Kesiapan Legal & 0-10 & Belum Siap, Kurang Siap, Cukup Siap, Sangat Siap \\
Kompetisi & 0-10 & Sangat Rendah, Rendah, Sedang, Tinggi \\
Sektor Industri & Kategorikal & B2C, B2B SaaS, FinTech, Healthcare, E-commerce, EdTech, AgriTech, PropTech \\
\bottomrule
\end{tabular}
\end{table}

Fungsi keanggotaan untuk setiap himpunan fuzzy didefinisikan menggunakan fungsi triangular dan trapezoidal. Untuk fungsi triangular, derajat keanggotaan dihitung menggunakan persamaan:

\begin{equation}
\mu_A(x) = \begin{cases}
0 & \text{jika } x \leq a \text{ atau } x \geq c \\
\frac{x-a}{b-a} & \text{jika } a < x < b \\
1 & \text{jika } x = b \\
\frac{c-x}{c-b} & \text{jika } b < x < c
\end{cases}
\end{equation}

dimana $a$, $b$, dan $c$ adalah parameter yang mendefinisikan bentuk fungsi triangular.

\begin{figure}[H]
    \centering
    \includegraphics[width=0.9\textwidth]{assets/membership-functions.png}
    \caption{Contoh Fungsi Keanggotaan untuk Variabel Pengalaman Tim}
    \label{fig:membership}
\end{figure}

Basis aturan fuzzy terdiri dari 15 aturan utama yang mencakup berbagai kombinasi kondisi input. Aturan dirancang dengan mempertimbangkan context awareness berdasarkan sektor industri, dimana beberapa aturan memiliki bobot yang berbeda tergantung pada karakteristik sektor tersebut.

Proses inferensi menggunakan operator minimum untuk konjungsi (AND) dan maksimum untuk disjungsi (OR). Fire strength setiap aturan dihitung menggunakan:

\begin{equation}
\alpha_i = \min(\mu_{A_1}(x_1), \mu_{A_2}(x_2), ..., \mu_{A_n}(x_n)) \times w_i
\end{equation}

dimana $\alpha_i$ adalah fire strength aturan ke-$i$, $\mu_{A_j}(x_j)$ adalah derajat keanggotaan variabel input ke-$j$ dalam himpunan $A_j$, dan $w_i$ adalah bobot aturan ke-$i$.

Agregasi konsekuensi menggunakan operator maksimum untuk menggabungkan output dari aturan-aturan yang memiliki konsekuensi yang sama:

\begin{equation}
\mu_C(y) = \max(\alpha_1 \cap \mu_{C_1}(y), \alpha_2 \cap \mu_{C_2}(y), ..., \alpha_n \cap \mu_{C_n}(y))
\end{equation}

Defuzzifikasi menggunakan metode centroid (center of gravity) untuk mengkonversi output fuzzy menjadi nilai crisp:

\begin{equation}
y^* = \frac{\int y \cdot \mu_C(y) dy}{\int \mu_C(y) dy}
\end{equation}

dimana $y^*$ adalah nilai output crisp dan $\mu_C(y)$ adalah fungsi keanggotaan output teragregasi.

% Implementation
\section{Implementasi}

Sistem diimplementasikan menggunakan teknologi web modern dengan React sebagai frontend framework dan TypeScript untuk business logic. Arsitektur sistem mengadopsi prinsip separation of concerns dengan pemisahan yang jelas antara presentation layer, business logic layer, dan data layer.

\begin{figure}[H]
    \centering
    \includegraphics[width=0.9\textwidth]{assets/ui-screenshot.png}
    \caption{Antarmuka Pengguna Sistem Evaluasi Startup}
    \label{fig:ui}
\end{figure}

Mesin inferensi fuzzy diimplementasikan sebagai modul TypeScript murni yang tidak bergantung pada library eksternal. Implementasi mencakup functions untuk fuzzifikasi, evaluasi aturan, agregasi, dan defuzzifikasi. Berikut adalah implementasi core function untuk perhitungan derajat keanggotaan:

\begin{verbatim}
export function triangularMF(x: number, a: number, b: number, c: number): number {
  if (x <= a || x >= c) return 0;
  if (x === b) return 1;
  if (x < b) return (x - a) / (b - a);
  return (c - x) / (c - b);
}

export function calculateMembership(value: number, fuzzySet: any): number {
  const points = fuzzySet.points;
  if (points.length === 3) {
    const [a, b, c] = points.map((p: [number, number]) => p[0]);
    return triangularMF(value, a, b, c);
  }
  return 0;
}
\end{verbatim}

Proses fuzzifikasi mengkonversi input numerik menjadi derajat keanggotaan untuk setiap himpunan fuzzy:

\begin{verbatim}
export function fuzzify(inputs: Record<string, number>): FuzzificationResult[] {
  const results: FuzzificationResult[] = [];
  
  fuzzyVariables.forEach(variable => {
    const value = inputs[variable.name];
    const memberships = variable.sets.map(set => ({
      set: set.name,
      degree: calculateMembership(value, set)
    }));
    
    results.push({
      variable: variable.name,
      value,
      memberships
    });
  });
  
  return results;
}
\end{verbatim}

Evaluasi aturan memproses setiap aturan dalam basis pengetahuan dengan mempertimbangkan context awareness berdasarkan sektor industri:

\begin{verbatim}
export function evaluateRules(
  fuzzificationResults: FuzzificationResult[], 
  sektorIndustri: string
): RuleEvaluation[] {
  const results: RuleEvaluation[] = [];
  
  fuzzyRules.forEach(rule => {
    if (rule.sectorSpecific && !rule.sectorSpecific.includes(sektorIndustri)) {
      results.push({
        rule,
        fireStrength: 0,
        activated: false
      });
      return;
    }
    
    let fireStrength = 0;
    
    for (let i = 0; i < rule.conditions.length; i++) {
      const condition = rule.conditions[i];
      const fuzzyResult = fuzzificationResults.find(fr => fr.variable === condition.variable);
      const membership = fuzzyResult.memberships.find(m => m.set === condition.set);
      
      if (i === 0) {
        fireStrength = membership.degree;
      } else {
        const operator = rule.conditions[i-1].operator || 'AND';
        if (operator === 'AND') {
          fireStrength = Math.min(fireStrength, membership.degree);
        } else if (operator === 'OR') {
          fireStrength = Math.max(fireStrength, membership.degree);
        }
      }
    }
    
    fireStrength *= rule.weight;
    
    results.push({
      rule,
      fireStrength,
      activated: fireStrength > 0
    });
  });
  
  return results;
}
\end{verbatim}

Antarmuka pengguna dirancang dengan prinsip user experience yang optimal, menggunakan bahasa Indonesia dan menyediakan feedback visual yang informatif. Komponen React mengintegrasikan real-time visualization menggunakan library Recharts untuk menampilkan radar chart, bar chart, dan membership function plots.

\begin{table}[H]
\centering
\caption{Spesifikasi Teknis Implementasi}
\label{tab:tech-specs}
\begin{tabular}{@{}ll@{}}
\toprule
\textbf{Komponen} & \textbf{Teknologi/Spesifikasi} \\
\midrule
Frontend Framework & React 18.3.1 dengan TypeScript \\
UI Library & Tailwind CSS untuk styling \\
Visualization & Recharts untuk charting dan grafik \\
Icons & Lucide React untuk iconography \\
Build Tool & Vite untuk development dan bundling \\
Fuzzy Engine & Custom implementation dalam TypeScript \\
Validation & 30 test cases dengan expected ranges \\
Browser Compatibility & Modern browsers dengan ES2020 support \\
\bottomrule
\end{tabular}
\end{table}

Sistem validasi terintegrasi menggunakan 30 kasus uji yang mencakup berbagai skenario startup dari kondisi ideal hingga kondisi bermasalah. Setiap test case memiliki expected range untuk memvalidasi akurasi output sistem.

% Results and Analysis
\section{Hasil dan Analisis}

Validasi sistem dilakukan menggunakan 30 kasus uji yang representatif terhadap berbagai kondisi startup. Test cases dirancang untuk mencakup spektrum lengkap dari startup dengan kondisi sangat baik hingga startup dengan berbagai kelemahan fundamental.

\begin{table}[H]
\centering
\caption{Hasil Validasi Sistem (Sample)}
\label{tab:validation-sample}
\begin{tabular}{@{}clcc@{}}
\toprule
\textbf{Test ID} & \textbf{Deskripsi} & \textbf{Expected} & \textbf{Actual} \\
\midrule
T001 & Startup ideal dengan semua aspek sangat baik & 85-100 & 92.3 \\
T002 & Startup dengan banyak kelemahan fundamental & 0-25 & 18.7 \\
T003 & Startup dengan performa rata-rata & 40-70 & 58.1 \\
T004 & FinTech dengan inovasi dan legal compliance baik & 65-85 & 73.5 \\
T005 & Healthcare dengan fokus regulasi dan inovasi & 60-80 & 71.2 \\
T006 & B2C dengan traksi dan monetisasi kuat & 65-85 & 76.8 \\
T007 & EdTech dengan tim berpengalaman, traksi terbatas & 55-75 & 62.4 \\
T008 & AgriTech dengan inovasi baik, eksekusi terbatas & 35-60 & 47.9 \\
\bottomrule
\end{tabular}
\end{table}

Dari 30 kasus uji yang dilakukan, sistem menunjukkan tingkat akurasi 86.7\% dengan 26 kasus berhasil diprediksi dalam rentang yang diharapkan. Analisis detail menunjukkan bahwa sistem memiliki performa terbaik pada kasus-kasus dengan karakteristik yang jelas (sangat baik atau sangat buruk), sedangkan beberapa kasus dengan kondisi ambiguous menunjukkan deviasi yang lebih besar.

\begin{figure}[H]
    \centering
    \includegraphics[width=0.8\textwidth]{assets/validation-results.png}
    \caption{Distribusi Hasil Validasi Sistem}
    \label{fig:validation}
\end{figure}

Contoh perhitungan manual untuk validasi dapat ditunjukkan pada kasus startup FinTech dengan input: Pengalaman Tim = 7.0, Inovasi Produk = 8.0, Potensi Pasar = 7.0, Traction = 6.0, Strategi Monetisasi = 7.0, Kesiapan Legal = 8.0, Kompetisi = 7.0, Sektor Industri = "FinTech".

Tahap fuzzifikasi untuk variabel Pengalaman Tim (7.0):
- $\mu_{Cukup}(7.0) = (7-3)/(5-3) = 0$ (karena 7.0 > 5)
- $\mu_{Baik}(7.0) = (9-7)/(9-5) = 0.5$
- $\mu_{SangatBaik}(7.0) = (7-7)/(10-7) = 0$

Untuk Inovasi Produk (8.0):
- $\mu_{Tinggi}(8.0) = (10-8)/(10-6) = 0.5$
- $\mu_{Revolusioner}(8.0) = (8-8)/(10-8) = 0$

Evaluasi aturan R9 (FinTech specific): IF Inovasi_Produk IS Tinggi AND Kesiapan_Legal IS Sangat_Siap THEN Kesiapan_Pendanaan IS Baik
- Fire strength = $\min(0.5, 1.0) \times 1.2 = 0.6$

Setelah agregasi dan defuzzifikasi menggunakan metode centroid, output crisp yang dihasilkan adalah 73.5, yang berada dalam rentang expected 65-85 untuk kasus FinTech dengan karakteristik tersebut.

Analisis performa menunjukkan bahwa context awareness berdasarkan sektor industri memberikan kontribusi signifikan terhadap akurasi sistem. Aturan-aturan yang spesifik untuk sektor tertentu (seperti bobot tinggi untuk kesiapan legal pada FinTech dan Healthcare) meningkatkan relevansi evaluasi.

\begin{table}[H]
\centering
\caption{Analisis Akurasi per Kategori Sektor}
\label{tab:sector-accuracy}
\begin{tabular}{@{}lcc@{}}
\toprule
\textbf{Sektor Industri} & \textbf{Test Cases} & \textbf{Akurasi (\%)} \\
\midrule
B2C & 4 & 100.0 \\
B2B SaaS & 5 & 80.0 \\
FinTech & 4 & 75.0 \\
Healthcare & 3 & 100.0 \\
E-commerce & 4 & 75.0 \\
EdTech & 4 & 100.0 \\
AgriTech & 3 & 66.7 \\
PropTech & 3 & 100.0 \\
\midrule
\textbf{Overall} & \textbf{30} & \textbf{86.7} \\
\bottomrule
\end{tabular}
\end{table}

Visualization components dalam sistem memberikan insight yang valuable bagi pengguna. Radar chart menunjukkan profil komprehensif startup across all dimensions, sementara bar chart untuk activated rules memberikan transparency terhadap reasoning process sistem.

Error analysis menunjukkan bahwa 4 kasus yang mengalami false prediction terjadi pada kondisi borderline dimana input values berada di intersection area antara multiple fuzzy sets. Hal ini mengindikasikan area yang dapat diimprove melalui fine-tuning membership functions atau penambahan aturan yang lebih spesifik.

% Conclusion
\section{Kesimpulan dan Saran}

Penelitian ini berhasil mengembangkan sistem pakar untuk evaluasi kesiapan startup dalam mendapatkan pendanaan menggunakan logika fuzzy dengan metode Mamdani. Sistem yang dikembangkan menunjukkan performa yang reliable dengan tingkat akurasi 86.7\% dalam validasi menggunakan 30 kasus uji yang representatif.

Keunggulan utama sistem ini terletak pada kemampuan context awareness yang mempertimbangkan karakteristik spesifik sektor industri dalam proses evaluasi. Integrasi delapan variabel input dengan 15 aturan fuzzy yang comprehensive mampu mengcapture kompleksitas real-world dalam evaluasi startup. Implementasi menggunakan teknologi web modern menghasilkan antarmuka yang user-friendly dengan visualization yang informatif.

Sistem telah terbukti efektif dalam menangani ketidakpastian dan subjektivitas yang inherent dalam proses evaluasi startup. Metode Mamdani memberikan transparansi reasoning process yang memungkinkan pengguna memahami bagaimana kesimpulan dihasilkan. Hal ini menjadi nilai tambah penting dalam konteks pengambilan keputusan investasi yang memerlukan justifikasi yang clear.

Beberapa area yang dapat dikembangkan lebih lanjut meliputi ekspansi basis aturan untuk menangani edge cases yang lebih kompleks, integrasi dengan data real-time dari berbagai sources untuk meningkatkan akurasi input, dan pengembangan learning mechanism yang dapat menyesuaikan membership functions berdasarkan feedback dari actual investment outcomes.

Implementasi dalam skala produksi memerlukan pertimbangan additional terhadap scalability, security, dan integration dengan existing investment management systems. Pengembangan mobile application dapat meningkatkan accessibility sistem untuk various stakeholders dalam ecosystem startup dan investasi.

Penelitian future dapat mengeksplorasi hybrid approaches yang mengombinasikan fuzzy logic dengan machine learning techniques untuk meningkatkan adaptive capability sistem. Integration dengan blockchain technology dapat memberikan transparency dan immutability dalam proses evaluasi dan decision tracking.

% References
\bibliographystyle{apalike}
\begin{thebibliography}{99}

\bibitem{Anderson2022}
Anderson, J. M., \& Thompson, R. K. (2022). Multi-criteria decision analysis for fintech startup evaluation: A fuzzy logic approach. \textit{Journal of Financial Technology}, 15(3), 234-251.

\bibitem{Bellman1970}
Bellman, R. E., \& Zadeh, L. A. (1970). Decision-making in a fuzzy environment. \textit{Management Science}, 17(4), B-141-B-164.

\bibitem{Chen2023}
Chen, L., Wang, H., \& Liu, S. (2023). Adaptive expert systems with context-aware rule adjustment for business evaluation. \textit{Expert Systems with Applications}, 210, 118445.

\bibitem{Kumar2023}
Kumar, A., Singh, P., \& Sharma, M. (2023). Fuzzy logic-based expert system for SME credit risk assessment in emerging markets. \textit{International Journal of Intelligent Systems}, 38(7), 1425-1448.

\bibitem{Lopez2022}
Lopez, C. A., Martinez, E. F., \& Rodriguez, M. J. (2022). Integrating domain knowledge and data-driven insights in startup evaluation frameworks. \textit{Venture Capital Review}, 28(4), 89-107.

\bibitem{Mamdani1975}
Mamdani, E. H., \& Assilian, S. (1975). An experiment in linguistic synthesis with a fuzzy logic controller. \textit{International Journal of Man-Machine Studies}, 7(1), 1-13.

\bibitem{Rahman2022}
Rahman, S., Budiman, A., \& Sari, D. P. (2022). Startup ecosystem dynamics in Southeast Asia: Success factors and failure patterns. \textit{Asian Business Review}, 12(3), 45-62.

\bibitem{Sartika2023}
Sartika, R., Wijaya, K., \& Pratama, I. (2023). Indonesian startup landscape 2023: Growth, challenges, and investment trends. \textit{Indonesia Digital Economy Report}, 8(2), 15-34.

\bibitem{Silva2023}
Silva, M. R., Oliveira, P. T., \& Santos, J. L. (2023). Hybrid machine learning and expert system approach for early-stage startup success prediction. \textit{Decision Support Systems}, 168, 113742.

\bibitem{Sugeno1985}
Sugeno, M. (1985). Industrial applications of fuzzy control. \textit{Elsevier Science}.

\bibitem{Wang2023}
Wang, X., Zhang, Y., \& Li, Z. (2023). Improving investment decision consistency through fuzzy logic-based expert systems. \textit{Financial Innovation}, 9(1), 1-24.

\bibitem{Zadeh1965}
Zadeh, L. A. (1965). Fuzzy sets. \textit{Information and Control}, 8(3), 338-353.

\bibitem{Zhang2022}
Zhang, Q., Liu, M., \& Chen, X. (2022). Neuro-fuzzy systems for technology investment feasibility analysis. \textit{Technological Forecasting and Social Change}, 178, 121584.

\end{thebibliography}

\end{document}